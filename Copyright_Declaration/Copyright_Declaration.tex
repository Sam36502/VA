\documentclass{article}

\begin{document}

\section{Bedingungen}
\begin{itemize}
    \item Ich kennzeichne klar, wo ich wörtlich zitiere, und weise auch darauf hin, wenn ich Erkenntnisse
    anderer umschreibe oder zusammenfasse. Damit ermögliche ich den Lesern und Leserinnen,
    die Herkunft und Qualität der von mir benutzten Information richtig einzuschätzen.

    \item Ich achte darauf, dass die Informationen, die ich von anderen bezogen habe, klar von meinen eigenen
    Überlegungen und Folgerungen unterschieden werden können.
    Erst dadurch wird auch meine eigene Leistung richtig einschätzbar.

    \item Ich achte darauf, dass meine bibliographischen Angaben so genau sind,
    dass sie den Lesern und Leserinnen das Auffinden der Quellen ermöglichen.

    \item Auch die aus dem Internet bezogene wissenschaftliche Information belege ich klar
    nach Herkunft von Texten und Bildern mit entsprechenden Internet-Adressen.
\end{itemize}

\section{Unterschrift}
\hfill \break
\hfill \break
\noindent\begin{tabular}{ll}
\makebox[2.5in]{\hrulefill} & \makebox[2.5in]{\hrulefill}\\
Samuel Pearce & Datum\\
\end{tabular}

\end{document}