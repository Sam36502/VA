\documentclass{article}
%\usepackage{fontspec}
\usepackage{tipa}
\usepackage{natbib}
\usepackage[utf8]{inputenc}

\begin{document}

%\newfontfamily{\lmln}[
%    Path=../fonts/,
%    UprightFont=Lumlun Sans Regular.ttf
%]{Lumlun Sans}
%
%\newcommand{\ortho}[1]{{\lmln\Large{\raggedright#1}\par}}

\begingroup
\centering
\vfill
\Large{Eine Vertiefungsarbeit über}\\
\Huge{KÜNSTLICHE SPRACHEN}\\
\huge{und}\\
\huge{MINIMALISTISCHE SPRACHEN}\\
\large{für die}\\
\Large{Technische Berufsschule Zürich}\\
\vspace{3cm}
\Large{von Samuel Pearce}\\
\vfill\null
\endgroup
\thispagestyle{empty}

\tableofcontents
\pagebreak

%
% Things to mention:
% - Underlying theory -> short summary about UG
% - Language creation process
% - Experimentation process
% - Text Results
% - Review of language capacity
% - Orthography Creation -> font creation
% - Conclusion about language capacity
% 

\section{Einführung}
\cite{Stevens72}



\section{Kurze Erklärung der UG Theorie}



\section{Prozess der Spracherstellung}


\section{Prozess der Sprachprüfung}


\section{Resultate des Experiments}



\section{Kapazität der Sprache }


\bibliography{Bibliography}
\bibliographystyle{plain}

\end{document}