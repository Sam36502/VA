\documentclass{article}
\usepackage[ngerman]{babel}
%\usepackage{fontspec}
\usepackage{tipa}
\usepackage{natbib}
\usepackage[utf8]{inputenc}

\begin{document}

%\newfontfamily{\lmln}[
%    Path=../fonts/,
%    UprightFont=Lumlun Sans Regular.ttf
%]{Lumlun Sans}
%
%\newcommand{\ortho}[1]{{\lmln\Large{\raggedright#1}\par}}

\begingroup
\centering
\vfill
\Large{Eine Vertiefungsarbeit über}\\
\Huge{KÜNSTLICHE SPRACHEN}\\
\huge{und}\\
\huge{MINIMALISTISCHE SPRACHEN}\\
\large{für die}\\
\Large{Technische Berufsschule Zürich}\\
\vspace{3cm}
\Large{von Samuel Pearce}\\
\vfill\null
\endgroup
\thispagestyle{empty}

\tableofcontents
\pagebreak

%
% Things to mention:
% - Underlying theory -> short summary about UG
% - Language creation process
% - Experimentation process
% - Text Results
% - Review of language capacity
% - Orthography Creation -> font creation
% - Conclusion about language capacity
% 

\begin{abstract}
    Im Laufe meiner VA habe ich versucht, die Beziehung zwischen dem Umfang einer Sprache
    (d.h. der Anzahl der allgemein gebräuchlichen Wörter und der Komplexität ihrer Grammatik)
    und ihrer Verwendbarkeit im Alltag zu entdecken und besser zu verstehen.
    Zu diesem Zweck habe ich eine Weile damit verbracht, meine eigene Sprache von Grund auf zu
    entwickeln und einige Texte in diese Sprache zu übersetzen. Dann habe ich die Texte an meine
    Freunde weitergegeben, die versucht haben, sie ins Deutsche zurück zu übersetzen.
    So konnte ich feststellen, wie schwer die Sprache zu verstehen ist.
    % TODO: Add conclusions after project.
\end{abstract}



\section{Einführung}
\cite{Stevens72}



\section{Kurze Erklärung der UG Theorie}



\section{Prozess der Spracherstellung}



\section{Prozess der Sprachprüfung}



\section{Resultate des Experiments}



\section{Kapazität der Sprache}



\section{Prozess der Erstellung der Orthografie}



\section{Schlussfolgerungen zum Verhältnis zwischen Sprachumfang und Nutzbarkeit}



\bibliography{Bibliography}
\bibliographystyle{plain}

\end{document}