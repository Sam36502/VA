\documentclass[a4paper,10pt]{article}
\usepackage[utf8]{inputenc}

%opening
\title{Language Creation Journal}
\author{Samuel Pearce}

\begin{document}

\maketitle

\tableofcontents

\pagebreak

\section{Introduction}
When I set out to create this language, its main goal was to be as small as possible,
in the sense that there were very few vocabulary to learn. At first, I figured I might
attempt this by simply limiting the number of words to an arbitrarily small set and then
see how useful the language was, and what might need to be added to make it useful in
daily life. Soon though, I realised that making it nearer to a polysynthetic language
with many uniform morphological derivation strategies, I could define an extremely limited
set of roots and then expand them easily with well defined rules. This is the base idea
which influenced much of the grammatical decision making.

\section{Log Entries}

\subsection{Tuesday 5\textsuperscript{th} of October}
This is the day I first began developing the language. I had already had a few thoughts
about what sounds I wanted to use and the general structure of the grammar, but this was
the first day I put it into writing. I began by outlining which sounds I wanted to use.
I eventually settled on an extremely regular set of twelve relatively easy-to-pronounce
consonants (with suitable allophones, in case) and seven vowels laid out in three pairs
of rounded-unrounded vowels and the schwa. I set out a romanisation which would emphasize
the paired nature of the sounds and set to work coining the first couple words to test
their various inflections. Here I first laid out the biconsonantal root system where the
vowel corresponds to the part of speech (Noun, Verb, Adjective) and defined how the
roundness-umlaut changes each of these. I also defined the cases and their suffixes as well
as which tenses would be available and their respective tenses.

\subsection{Friday 8\textsuperscript{th} of October}
During the week I made a few additions, such as the basic person pronouns and a couple new
roots to make more test sentences and I thought about what other suffixes or prefixes I
could add to allow for more derivations. I also began defining how to ask questions.

\subsection{Saturday 9\textsuperscript{th} of October}
On this weekend, my two friends who were volunteering to help me test out the language's
applicability visited for a couple days during which we made a few changes to the language.
The vowels all shifted somewhat to make them easier to produce and distinguish for us. I
taught them the basics of the noun classes and tenses and we made a few sentences to test
their knowledge and to see if they had any suggestions. We solidified how to ask for
information as well as binary questions. 

\end{document}