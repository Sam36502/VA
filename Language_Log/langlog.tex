\documentclass[a4paper,10pt]{article}
\usepackage[utf8]{inputenc}

%opening
\title{Language Creation Journal}
\author{Samuel Pearce}

\begin{document}

\maketitle

\tableofcontents

\pagebreak

\section{Introduction}
When I set out to create this language, its main goal was to be as small as possible,
in the sense that there were very few vocabulary to learn. At first, I figured I might
attempt this by simply limiting the number of words to an arbitrarily small set and then
see how useful the language was, and what might need to be added to make it useful in
daily life. Soon though, I realised that making it nearer to a polysynthetic language
with many uniform morphological derivation strategies, I could define an extremely limited
set of roots and then expand them easily with well defined rules. This is the base idea
which influenced much of the grammatical decision making.

\section{Log Entries}

\subsection{Tuesday 5\textsuperscript{th} of October}
This is the day I first began developing the language. I had already had a few thoughts
about what sounds I wanted to use and the general structure of the grammar, but this was
the first day I put it into writing. I began by outlining which sounds I wanted to use.
I eventually settled on an extremely regular set of twelve relatively easy-to-pronounce
consonants (with suitable allophones, in case) and seven vowels laid out in three pairs
of rounded-unrounded vowels and the schwa. I set out a romanisation which would emphasize
the paired nature of the sounds and set to work coining the first couple words to test
their various inflections. Here I first laid out the biconsonantal root system where the
vowel corresponds to the part of speech (Noun, Verb, Adjective) and defined how the
roundness-umlaut changes each of these. I also defined the cases and their suffixes as well
as which tenses would be available and their respective tenses.

\subsection{Friday 8\textsuperscript{th} of October}
During the week I made a few additions, such as the basic person pronouns and a couple new
roots to make more test sentences and I thought about what other suffixes or prefixes I
could add to allow for more derivations. I also began defining how to ask questions.

\subsection{Saturday 9\textsuperscript{th} of October}
On this weekend, my two friends who were volunteering to help me test out the language's
applicability visited for a couple days during which we made a few changes to the language.
The vowels all shifted somewhat to make them easier to produce and distinguish for us. I
taught them the basics of the noun classes and tenses and we made a few sentences to test
their knowledge and to see if they had any suggestions. We solidified how to ask for
information as well as binary questions. We also defined a few general inflections, such
as the augmentitive and diminutive prefixes which are very handy for deriving new words.
We spent a while discussing whether or not to define a way to set the aspect of the verbs,
but determined it not really useful enough for now. I also wanted to get a foundation of
the writing system defined and asked my volunteers for some suggestions of what \ae sthetic
direction to move in.

\subsection{Thursday 21\textsuperscript{st} of October}
I had some free time at work and spent a good portion of my afternoon documenting all the
features of the language and ensuring that they were unambiguous. I went through the whole
adjective inflection system, defined how to mark noun definitiveness, and even got the
basics of different clauses worked out, so that conditional sentences and conjunctions
could be used. Though these weren't fully fleshed-out by the end of the day, I could
already make some pretty complex sentences like: 

\begin{center}
    ``Up kiñ kuñfuke utem, up nif muk, ut ñik taf puxe upem.'' \\
    Which means:\\
    ``I give you some money, because I want you to get me good food.''
\end{center}

Though the phrases were still quite poorly defined, and this particular example made the
need for unambiguous prepositions very clear.

\subsection{Sunday 24\textsuperscript{th} of October}
On this day, I spent some more time working on the documentation of the language, and
also finally decided to remove the ``eng'' sound from the language as it was too difficult
to differentiate, especially in onset position. I also defined how prepositions were going
to work after realising the other day how ambiguous they could be. Additionally, I spent
some more time refining the glyph designs for the writing system and began attempting to
create a font for it.

\end{document}