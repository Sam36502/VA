\documentclass[a4paper,10pt]{article}
\usepackage[utf8]{inputenc}

%opening
\title{Language Creation Journal}
\author{Samuel Pearce}

\begin{document}

\maketitle

\tableofcontents

\pagebreak

\section{Introduction}
When I set out to create this language, its main goal was to be as small as possible,
in the sense that there were very few vocabulary to learn. At first, I figured I might
attempt this by simply limiting the number of words to an arbitrarily small set and then
see how useful the language was, and what might need to be added to make it useful in
daily life. Soon though, I realised that making it nearer to a polysynthetic language
with many uniform morphological derivation strategies, I could define an extremely limited
set of roots and then expand them easily with well defined rules. This is the base idea
which influenced much of the grammatical decision making.

\section{Log Entries}

\subsection{Tuesday 5\textsuperscript{th} of October}
This is the day I first began developing the language. I had already had a few thoughts
about what sounds I wanted to use and the general structure of the grammar, but this was
the first day I put it into writing. I began by outlining which sounds I wanted to use.
I eventually settled on an extremely regular set of twelve relatively easy-to-pronounce
consonants (with suitable allophones, in case) and seven vowels laid out in three pairs
of rounded-unrounded vowels and the schwa. I set out a romanisation which would emphasize
the paired nature of the sounds and set to work coining the first couple words to test
their various inflections. Here I first laid out the biconsonantal root system where the
vowel corresponds to the part of speech (Noun, Verb, Adjective) and defined how the
roundness-umlaut changes each of these. I also defined the cases and their suffixes as well
as which tenses would be available and their respective tenses.

\subsection{Friday 8\textsuperscript{th} of October}
During the week I made a few additions, such as the basic personal pronouns and a couple new
roots to make more test sentences and I thought about what other suffixes or prefixes I
could add to allow for more derivations. I also began defining how to ask questions.

\subsection{Saturday 9\textsuperscript{th} of October}
On this weekend, my two friends who were volunteering to help me test out the language's
applicability visited for a couple days during which we made a few changes to the language.
The vowels all shifted somewhat to make them easier to produce and distinguish for us. I
taught them the basics of the noun classes and tenses and we made a few sentences to test
their knowledge and to see if they had any suggestions. We solidified how to ask for
information as well as binary questions. We also defined a few general inflections, such
as the augmentitive and diminutive prefixes which are very handy for deriving new words.
We spent a while discussing whether or not to define a way to set the aspect of the verbs,
but determined it not really useful enough for now. I also wanted to get a foundation of
the writing system defined and asked my volunteers for some suggestions of what \ae sthetic
direction to move in.

\subsection{Thursday 21\textsuperscript{st} of October}
I had some free time at work and spent a good portion of my afternoon documenting all the
features of the language and ensuring that they were unambiguous. I went through the whole
adjective inflection system, defined how to mark noun definitiveness, and even got the
basics of different clauses worked out, so that conditional sentences and conjunctions
could be used. Though these weren't fully fleshed-out by the end of the day, I could
already make some pretty complex sentences like: 

\begin{center}
    ``Up kiñ kuñfuke utem, up nif muk, ut ñik taf puxe upem.'' \\
    Which means:\\
    ``I give you some money, because I want you to get me good food.''
\end{center}

Though the phrases were still quite poorly defined, and this particular example made the
need for unambiguous prepositions very clear.

\subsection{Sunday 24\textsuperscript{th} of October}
On this day, I spent some more time working on the documentation of the language, and
also finally decided to remove the ``eng'' sound from the language as it was too difficult
to differentiate, especially in onset position. I also defined how prepositions were going
to work after realising the other day how ambiguous they could be. Additionally, I spent
some more time refining the glyph designs for the writing system and began attempting to
create a font for it.

\subsection{Friday 30\textsuperscript{th} of October}
Here, I had some overtime built up and had to get rid of it by the end of the month, so ---
given I didn't have any tasks for my at work --- I was allowed to take the rest of the
afternoon to work on my VA. During this time, I continued transferring temporary notes which
had been tested and worked in the language to the grammar document. This mainly included
newly coined words. I also tried translating ``The North Wind \& The Sun'', a commonly used
fable for conlang translations, which lead to the dicovery of many problematic ambiguities.
I didn't manage to get very far in the translation as I spent most of the time given working
on ironing out these troublesome concepts.

\subsection{Monday 1\textsuperscript{st} of November}
Given my next task at work wasn't ready for me to work on, and there weren't any other tasks
available, my boss allowed me to work on the VA for a period instead of work. I used this
time to go through all the areas that were unclear or not yet documented and I tested them
out while adding them to the grammar document so I could at least have it concretely
documented. The ideas I went through included basic conjunctions with some extra prepositions,

\subsection{Tuesday 2\textsuperscript{nd} of November}
On this day, I tried making some more progress with ``The North Wind \& The Sun''. Most of time
was spent trying to find unambiguous ways of representing more complex topics than the few roots
alone could. I also was in contact with one of the volunteers to test out various compunds and
phrases to see how easy they were to understand.

\subsection{Wednesday 3\textsuperscript{rd} of November}
Here, I decided to reorganise the dictionary and work in the use of multiple small languages
for testing purposes finally. I realised that I could easily create a scale of dictionary
sizes with each one being larger than the last up to a certain maximum. This meant I could
test different language scales within the framework of a single language, while simultaneously
making it easier to learn by splitting it into easy-to-learn chunks, each building on the
previous. I began working on reducing the set of words I had to the lowest language mode, M0,
while using an online system to document them and how each is reduced to fewer, more ambiguous
roots, or more, clearer roots. For the numbering/naming convention, I decided each group of
four words would be represented by a number increasing steadily. So the the smallest version of
Sutlun with any roots is Sutlun M1 with 4 roots. Sutlun M2 has 8 roots, and so on until M25.
I also refined the set of grammar rules down to 7 simple rules, while also making the prefixes
less ambiguous. I tried teaching my parents some of the grammar to see how easy it was.

\subsection{Tuesday 9\textsuperscript{th} of November}
On this day, with free reign to add as many words as needed with the reassurance that I could
simply cut them out of the smaller lexicons, I resolved to finally finish translating ``The
North Wind \& Sun'' fable. I practiced writing it using my custom made writing system and
added any new words I'd coined to the dictionary I was keeping. Though this didn't mean the
language was complete, it was a landmark and a test of the grammar I'd made which I only needed
to lightly tweak to work with the story. Once I try translating a few more texts, I should be
able to vouch for the language's usability and can commence trying it out with my test subjects
and an even larger text, only maybe needing to add words as seems fit.

\subsection{Tuesday 16\textsuperscript{th} of November}
Here, I had to present the status of my VA to the class. Though I didn't actually have to
present it, I still had to prepare a presentation, which I did eventually actually present.
In it, I was able to look back and reflect on the progress so far. I was quite happy with
how much I'd done, but felt I'd been slacking in the past week and needed to invest some
more time this weekend. I began searching for a suitable larger text to translate as that
would be the next big step in the whole project.

\subsection{Saturday 27\textsuperscript{th} of November}
On the weekend, I decided I wanted to overhaul the grammar document, because I had not
updated it for a long time given that I wanted to develop the language more before writing
everything down to save time. So, now that the language was much more well-defined, I sat
down to go through the whole grammar and update it to reflect any new changes and to add
better example-glosses. One of the biggest challenges I ran into while translating and
adding new words was that I had made every word have an antonym, which helps greatly with
derivation, but makes it quite a pain when I have try to find a way to, for example, say
``table'' which only uses words which can have an opposite.

\subsection{Wednesday 1\textsuperscript{st} of December}
On this day, with the project deadline looming ever-closer, I focussed on just getting
the text translations complete so I could start working on the final document I'd actually
have to hand in. On this day, I worked on and finished the translation of ``The Tower of
Babel'' tale and was quite pleased to see that I only had to add two or three words. After
that, I considered the language done for now and sent the text to my two volunteers to try
to translate. I didn't tell them what the text was and left them to try their hand at it.
This way I would be able to get an idea of how ambiguous the words were.

\end{document}