\documentclass[a4paper,10pt]{article}
\usepackage[utf8]{inputenc}

%opening
\title{Project Description}
\author{Samuel Pearce}

\begin{document}

\maketitle

\tableofcontents

\pagebreak

\section{Problem Description}
In the course of this Advanced Paper, I intend to increase my knowledge on the scope of
language in relation to its complexity. I also intend to research language creation and
will create at least one language of my own to experiment with its scale and applicability.
The main problem I see is that I have never put my theoretical knowledge to the test in
the arena of language creation. I also lack the knowledge on microscopic languages and
wish to further my understanding in this field.

\section{Current State of Research}
Current experts in the field of language invention, such as David J. Peterson, creator of
the in-world languages for many popular modern media, like Dothraki from Game of Thrones
or Paul Frommer (inventor of Na'vi, the language from Avatar) have explored and paved the
path for language creation. Nowadays, learning the basics of linguistics through amateurishly
creating fantastical languages is --- in my personal opinion --- the best method to retain
such abstract and complex topics as Ergative-Absolutive Case and polypersonal agreement.

Existing research into language scope is fairly limited and mostly consists of common-sense
facts like the fact that a language with only one word is nearly entirely useless. Though
I have no exceptional expectation from my experimentation, I hope to at least document how
easily and richly a story could be translated into extremely compact languages.

\section{Questions to be Answered}
\begin{enumerate}
    \item How difficult is it to create a language?
    \item How small can a language be while still being applicable in most daily situations?
    \item How well can a short literary text be translated into such a small language?
    \item And (if time grants it), how difficult is it to create a writing system \& font
          conforming to an irregular orthography?
\end{enumerate}

\section{Method}
I propose to split this project into its two main constituent parts thusly:
I will begin by creating the language or languages and document the progress and grammar in
the first short paper. In the second, I'll document the language's effective usability by
testing it in a dialogue with volunteers and by translating a text into the language(s),
followed by a commentary on the effectiveness of the language in these situations.
Finally, if time permits, I would also like to enrich the language's appearance and learn
more about the development of writing systems and create a font which allows the language's
native orthography to written on computers.

\section{Required Materials}
For the proposed project should following materials be sufficient, all of which I already possess.
\begin{itemize}
    \item Mark Rosenfelder's Linguistic textbooks focusing on language creation:
    \begin{itemize}
        \item The Language Construction Kit
        \item Advanced Language Construction \&
        \item The Conlanger's Lexipedia
    \end{itemize}
    \item Various Online Resources
    \item Willing Volunteers for experimentation (Most likely my long-suffering friends)
    \item A computer equipped with text editor and \LaTeX \hspace{0.5mm} for document typesetting
    \item Open Source Font creation software (Birdfont)
\end{itemize}

\section{Time Plan}
I propose to divide the allotted 8 weeks as follows:
\begin{itemize}
    \item \textbf{Week 1:}
        \begin{itemize}
            \item Consulting with volunteers
            \item Language Development
            \item Logging Development Progress
        \end{itemize}
    \item \textbf{Week 2:}
        \begin{itemize}
            \item Language Development
            \item Logging Development Progress
        \end{itemize}
    \item \textbf{Week 3:}
        \begin{itemize}
            \item Language Development
            \item Logging Development Progress
        \end{itemize}
    \item \textbf{Week 4:}
        \begin{itemize}
            \item Finalize Languages
            \item Document Grammar and findings
            \item Dialogue with volunteers
        \end{itemize}
    \item \textbf{Week 5:}
        \begin{itemize}
            \item Text translation
            \item Document translation process
        \end{itemize}
    \pagebreak
    \item \textbf{Week 6:}
        \begin{itemize}
            \item Text translation
            \item Document translation process
        \end{itemize}
    \item \textbf{Week 7:}
        \begin{itemize}
            \item Complete Report on language efficacy and development process
            \item Possible Writing system drafting
        \end{itemize}
    \item \textbf{Week 8:}
        \begin{itemize}
            \item Finish Writing System and Font
            \item Perhaps write a small report on the script development
        \end{itemize}
\end{itemize}

\vspace{100mm}

\section{Signatures}
\hfill \break
\hfill \break
\noindent\begin{tabular}{ll}
\makebox[2.5in]{\hrulefill} & \makebox[2.5in]{\hrulefill}\\
Student & Date\\[8ex]% adds space between the two sets of signatures
\makebox[2.5in]{\hrulefill} & \makebox[2.5in]{\hrulefill}\\
Teacher & Date\\
\end{tabular}

\end{document}
