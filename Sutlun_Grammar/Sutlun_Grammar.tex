\documentclass{book}
\usepackage{tipa}
\usepackage{natbib}
\usepackage[utf8]{inputenc}


\newcommand{\entry}[4]{\textbf{#1} -- /\textipa{#2}/ \emph{#3} $\bullet$ #4\\}

\begin{document}

\begingroup
\centering
\vfill
\Huge{REFERENCE \\ GRAMMAR}\\
\huge{\&}\\
\Huge{DICTIONARY}\\
\huge{of}\\
\Huge{Sutlun}\\
\Large{Redo Title Page at some point...}\\
\large{By Samuel Pearce}\\
\vfill\null
\endgroup
\thispagestyle{empty}

\tableofcontents
\pagebreak

\section{Introduction}
Sutlun is a constructed language, or conlang, which was created in 2021 for a school
assignment as an experiment. The goal was to create a language which was as small and as
easy to learn as possible. It began as a series of small languages, each being smaller
than the last. I was looking for the point where a language becomes impractically small
and no longer useful in day-to-day life. Sutlun is the final result of this.



\part{Grammar}
\chapter{Phonology}
\section{Consonants}
\begin{center}
    \begin{tabular}{l|c|c|c}
                    & Bilabial          & Alveoalar  & Palatal \\
        \hline
        Nasal       & m                 & n         &  \\
        Plosive     & p                 & t         & k \\
        Fricative   & \textipa{F} [f]   & s         & x \\
        Liquid      & w                 & l         & j \\
    \end{tabular}
\end{center}

\section{Vowels}
\begin{center}
    \begin{tabular}{l|c|c|c}
                    & Front                         & Back \\
        \hline
        Close       & i $\cdot$ y                   & \textipa{W [2]} $\cdot$ u \\
        Middle      & e                             & \\
        Open        & \textipa{a} [\ae] $\cdot$ \oe & \\
    \end{tabular}
\end{center}

\section{Phonotactics}
In Sutlun, roots are bi-consonantal and the vowel determines what part of speech the word is.
For these root words, the consonant structure is \textbf{CVC} Where V is any vowel except /\textipa{@}/,
C is any consonant.

\section{Orthography}

\subsection{Romanisation}
The romanisation used might seem quite strange to an outside observer, but it was designed to
emphasize the duality of the main vowels (y, ú, a) with their rounded equivalents (ý, u, á) which
represents a change in meaning for the roots. Please note that the unrounded 'u' is marked, whereas
the other two unrounded main vowels aren't, this is due to front vowels typically being unrounded,
while back vowels are typically rounded \cite{Stevens72}. The more ``typical'' vowel is the ``default'' form,
while the less typical form is the inflected one. Though given that this might be difficult to understand
and not as easy to type as it is on a QWERTZ keyboard, a more phonetic alternative is also provided with
digraph alternatives to the diacritics used.

\begin{center}
    \begin{tabular}{|c|c|c|}
        \hline
        IPA & Rom. & Alt. \\
        \hline
        p           & p & p \\
        t           & t & t \\
        k           & k & k \\
        m           & m & m \\
        n           & n & n \\
        \textipa{F} & f & f \\
        s           & s & s \\
        x           & x & x \\
        \textipa{@} & e & e \\
        \hline
    \end{tabular}
    \begin{tabular}{|c|c|c|}
        \hline
        IPA & Rom. & Alt. \\
        \hline
        w           & w & w \\
        l           & l & l \\
        j           & j & j \\
        a           & a & a \\
        \oe         & á & ö/oe \\
        i           & y & i \\
        y           & ý & ü/ue \\
        \textipa{W} & ú & ue \\
        u           & u & uu/oo \\
        \hline
    \end{tabular}
\end{center}

\subsection{Writing System}
Given the rigidly structured syllables, I experimented with the idea of writing systems that used this
to their advantage for more regular and compact glyphs, but found this too complicated and received
feedback that confirmed this fear. So I decided to go for a simpler alphabetic system for the writing
system. I definitely wanted to make it a featural system though, because I had layed the phonemes out
in a systematic manner for this purpose.


\chapter{Morphology}
\section{Universal Inflections}
These are a few inflections (mostly prefixes) which can be applied to any root, no matter the
part of speech. Though these changes may not always yield a result that fully makes sense.

\subsection{Opposites}
You can form the opposite meaning of a word by rotating the root around it's vowel:

\begin{center}
    ``Taf'' $\rightarrow$ ``Good'' \\
    ``Fat'' $\rightarrow$ ``Bad''
\end{center}

\section{Nouns}
\subsection{Number}
In FSutlun, Nouns all have the ``u'' sound in the root which is unrounded for singular and rounded for plural.
For example:

\begin{center}
    ``Mun'' $\rightarrow$ ``a game'' \\
    ``Mún'' $\rightarrow$ ``many games''
\end{center}

\subsection{Case}
FSutlun has 4 grammatical cases which are all formed with a simple suffix according to the following table:

\begin{center}
    \begin{tabular}{|r|l|l|}
        \hline
        Case Name   & Suffix    & Example \\
        \hline
        Nominative  & -         & pux \\
        Accusative  & -e        & puxe \\
        Dative      & -em       & puxem \\
        Genitive    & -es       & puxes \\
        \hline
    \end{tabular}
\end{center}

\subsection{Definitiveness}
By default, nouns are indefinite and if they are definite, it can be parsed through context,
but if you wish to define a noun as being definite, you can give it the `-te' prefix.

\begin{center}
    ``Mun'' $\rightarrow$ ``a game'' \\
    ``Temun'' $\rightarrow$ ``the game''
\end{center}


\section{Verbs}
\subsection{Mood}
FSutlun has two verb moods: Indicative \& Imperative. These are also formed by the root-sound's roundness.
All Verbs use the ``y'' sound for their roots. ``Y'' is indicative, while ``ý'' is imperative:

\begin{center}
    ``ut kyn'' $\rightarrow$ ``You go.'' / ``You are going.'' \\
    ``ut kýn'' $\rightarrow$ ``You, go!''
\end{center}

\subsection{Tense}
FSutlun has 3 tenses which are all formed with a simple suffix according to the following table:

\begin{center}
    \begin{tabular}{|r|l|r|l|}
        \hline
        Tense Name  & Suffix    & Example   & Meaning \\
        \hline
        Past        & -et       & pixet     & ate, were eating \\
        Present     & - (-ef)   & pix       & eat, are eating \\
        Future      & -ej       & pixej     & will eat \\
        \hline
    \end{tabular}
\end{center}
The present tense is the default tense and needn't be marked, but if it is, it emphasizes that
the action is taking place now. E.g.:

\begin{center}
    ``ut kyn kumem?'' $\rightarrow$ ``Where are you going?'' \\
    ``ut kynef kumem?'' $\rightarrow$ ``Where are you going now?''
\end{center}


\section{Adjectives}
\subsection{Positive \& Superlative}
Adjectives in FSutlun all have the ``a'' sound in their root which is rounded to form the superlative
form of the adjective.

\begin{center}
    ``taf pux'' $\rightarrow$ ``good food'' \\
    ``táf pux'' $\rightarrow$ ``the best food''
\end{center}

Adjectives may also be used as the verb of the sentence meaning ``to be like <adjective>''. i.e.:

\begin{center}
    ``ut taf'' $\rightarrow$ ``You're good.'' \\
    ``mukmun mán'' $\rightarrow$ ``This game is the most fun.''
\end{center}

\subsection{Comparing}
To form the comparative of an adjective, you add the augmentative or diminutive prefix, depending
on whether you want the positive or negative form:

\begin{center}
    ``upes xul jutaf.'' $\rightarrow$ ``My house is better.'' \\
    ``ukes puxe ujtaf utes puxem'' $\rightarrow$ ``Their food is worse than your food.'' \\
    ``ut kenik juwas mun?'' $\rightarrow$ ``Did you get a newer game?''
\end{center}


\chapter{Syntax}

\chapter{Sentence Order}
to add: mik (to be) *may* be used as a copula, but is not required


\part{Lexicon}

\begin{center}
    \Huge{P}
\end{center}

{\Large\textbf{P-X} -- Food, Drink} \\
\emph{Antonym: ``Excretion, Expelling'' See X-P} \\
\entry{pux}{"pux}{n. sg.}{Food, Drink, an item of food, a meal}
\entry{púx}{"p2x}{n. pl.}{Food, many items of food/drink}
\entry{pyx}{"pix}{v. ind.}{to eat, to drink, to consume}
\entry{pýx}{"pyx}{v. imp.}{eat!, drink!, consume!}
\entry{pax}{"pax}{a. pos.}{delicious, tasty}
\entry{páx}{"p\oe x}{a. sup.}{most delicious}

\begin{center}
    \Huge{T}
\end{center}

{\Large\textbf{T-F} -- Good, Positive} \\
\emph{Antonym: ``Bad, Negative'' See F-T} \\
\entry{tuf}{"tuf}{n. sg.}{a good deed/thing, the concept of goodness}
\entry{túf}{"t2f}{n. pl.}{many good things}
\entry{tyf}{"tif}{v. ind.}{to improve, fix, better}
\entry{týf}{"tyf}{v. imp.}{fix!, improve!}
\entry{taf}{"taf}{a. pos.}{good, well}
\entry{táf}{"t\oe f}{a. sup.}{the best}

{\Large\textbf{T-S} -- Quiet, Still} \\
\emph{Antonym: ``Noise, Sound, Loud'' See S-T} \\
\entry{tus}{"tus}{n. sg.}{quiet, peace, stillness}
\entry{tús}{"t2s}{n. pl.}{much stillness, much peace}
\entry{tys}{"tis}{v. ind.}{to be quiet, calm down, make peace}
\entry{týs}{"tys}{v. imp.}{be quiet!, calm down!}
\entry{tas}{"tas}{a. pos.}{good, well}
\entry{tás}{"t\oe s}{a. sup.}{the best}

\begin{center}
    \Huge{K}
\end{center}

{\Large\textbf{K-M} -- Query, What?} \\
\emph{Antonym: ``Demonstrative, Thing, That'' See M-K} \\
\entry{kum}{"kum}{n. sg.}{what thing?}
\entry{kúm}{"k2m}{n. pl.}{what things?}
\entry{kym}{"kim}{v. ind.}{doing what?}
\entry{kým}{"kym}{v. imp.}{(special case) what are you doing?}
\entry{kam}{"kam}{a. pos.}{like what?}
\entry{kám}{"k\oe m}{a. sup.}{most like what?}
\entry{kem}{"kem}{excl.}{What!? (General indicator of confusion)}

{\Large\textbf{K-N} -- Go, Move, Give} \\
\emph{Antonym: ``Come, Bring'' See N-K} \\
\entry{kun}{"kun}{n. sg.}{a walk, a motion/movement, a journey}
\entry{kún}{"k2n}{n. pl.}{many walks, many motions/movements, many journeys}
\entry{kyn}{"kin}{v. ind.}{to walk, to move, to go}
\entry{kýn}{"kyn}{v. imp.}{walk!, move!, go!}
\entry{kan}{"kan}{a. pos.}{in motion, moving, going, living}
\entry{kán}{"k\oe n}{a. sup.}{moving the most, the most alive}
\entry{ken}{"ken}{prep.}{to, toward}

{\Large\textbf{K-F} -- Fantasy, Unreal, Fake} \\
\emph{Antonym: ``Thing, Real, Exist'' See F-K} \\
\entry{kuf}{"kuf}{n. sg.}{a fantasy, something fake/unreal}
\entry{kúf}{"k2f}{n. pl.}{many fantasies, unreal things}
\entry{kyf}{"kif}{v. ind.}{imagine, picture}
\entry{kýf}{"kyf}{v. imp.}{iamge!, picture!}
\entry{kaf}{"kaf}{a. pos.}{fake, unreal, fantasy, fictional}
\entry{káf}{"k\oe f}{a. sup.}{the most fake, unreal, fictional}

\begin{center}
    \Huge{M}
\end{center}

{\Large\textbf{M-K} -- Demonstrative, Thing} \\
\emph{Antonym: ``Query, What?'' See K-M} \\
\entry{muk}{"muk}{n. sg.}{this/that thing}
\entry{múk}{"m2k}{n. pl.}{these/those things}
\entry{myk}{"mik}{v. ind.}{doing this/that}
\entry{mýk}{"myk}{v. imp.}{do this/that!}
\entry{mak}{"mak}{a. pos.}{like this/that}
\entry{mák}{"m\oe k}{a. sup.}{most like this/that}

{\Large\textbf{M-N} -- Entertainment, Fun, Game} \\
\emph{Antonym: ``Work, Task, Boring'' See N-M} \\
\entry{mun}{"mun}{n. sg.}{a game, book, film, TV-show, play, etc.}
\entry{mún}{"m2n}{n. pl.}{many games, books, films, etc.}
\entry{myn}{"min}{v. ind.}{to play, entertain, relax}
\entry{mýn}{"myn}{v. imp.}{go play!, have fun!}
\entry{man}{"man}{a. pos.}{fun, entertaining}
\entry{mán}{"m\oe n}{a. sup.}{most fun, entertaining}

{\Large\textbf{M-L} -- Soft, Weak, Clay} \\
\emph{Antonym: ``Solid, Strong, Rock, Metal'' See L-M} \\
\entry{mul}{"mul}{n. sg.}{a soft thing, some clay, sand, paste, powder}
\entry{múl}{"m2l}{n. pl.}{many soft things, clay, sand, paste, powder}
\entry{myl}{"mil}{v. ind.}{to soften, weaken, mould}
\entry{mýl}{"myl}{v. imp.}{soften!, weaken!, mould!}
\entry{mal}{"mal}{a. pos.}{soft, weak}
\entry{mál}{"m\oe l}{a. sup.}{most soft, weak}

\begin{center}
    \Huge{N}
\end{center}

{\Large\textbf{N-M} -- Work, Task, Boring} \\
\emph{Antonym: ``Entertainment, Fun, Game'' See M-N} \\
\entry{num}{"num}{n. sg.}{a task, job, craft, skill}
\entry{núm}{"n2m}{n. pl.}{many tasks, jobs, crafts, skills}
\entry{nym}{"nim}{v. ind.}{to work, make, craft}
\entry{ným}{"nym}{v. imp.}{work!, maake!, craft!}
\entry{nam}{"nam}{a. pos.}{mandatory, boring, arduous}
\entry{nám}{"n\oe m}{a. sup.}{most boring, arduous}

{\Large\textbf{N-F} -- Like, Desire, Want} \\
\emph{Antonym: ``Dislike, Hate, Must'' See F-N} \\
\entry{nuf}{"nuf}{n. sg.}{a desire, want}
\entry{núf}{"n2f}{n. pl.}{many desires, wants}
\entry{nyf}{"nif}{v. ind.}{to like, desire, want}
\entry{nýf}{"nyf}{v. imp.}{like smth.! (special case:) exclamation of joy ``wow!'', ``yay!''}
\entry{naf}{"naf}{a. pos.}{nice , boring, arduous}
\entry{náf}{"n\oe f}{a. sup.}{most boring, arduous}

\begin{center}
    \Huge{F}
\end{center}

\begin{center}
    \Huge{S}
\end{center}

\begin{center}
    \Huge{X}
\end{center}

\begin{center}
    \Huge{W}
\end{center}

\begin{center}
    \Huge{L}
\end{center}

\begin{center}
    \Huge{J}
\end{center}

\begin{center}
    \Huge{I}
\end{center}

\begin{center}
    \Huge{U}
\end{center}

\begin{center}
    \Huge{A}
\end{center}

\begin{center}
    \Huge{E}
\end{center}

\bibliography{Bibliography}
\bibliographystyle{plain}

\end{document}
