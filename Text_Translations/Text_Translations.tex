\documentclass{book}
\usepackage{fontspec}
\usepackage{tipa}
\usepackage[utf8]{inputenc}

\begin{document}

\newfontfamily{\lmln}[
    Path=../fonts/,
    UprightFont=Lumlun Sans Regular.ttf
]{Lumlun Sans}

\newcommand{\ortho}[1]{{\lmln\Large{\raggedright#1}\par}}

\begingroup
\centering
\vfill
\Huge{VARIOUS}\\
\Huge{TEXTS \& PASSAGES}\\
\huge{translated into the}\\
\huge{SUTLUN LANGUAGE}\\
\large{in}\\
\Large{Some Lexicon Modes}\\
\vspace{3cm}
\Large{By Samuel Pearce}\\
\vfill\null
\endgroup
\thispagestyle{empty}

\tableofcontents
\pagebreak

\chapter{The North Wind \& Sun}
\section{English Text}

    The North Wind and the Sun disputed as to which was the most powerful,
    and agreed that he should be declared the victor who could first strip a wayfaring man of his clothes.
    The North Wind first tried his power and blew with all his might, but the keener his blasts,
    the closer the Traveler wrapped his cloak around him, until at last, resigning all hope of victory,
    the Wind called upon the Sun to see what he could do. The Sun suddenly shone out with all his warmth.
    The Traveler no sooner felt his genial rays than he took off one garment after another, and at last,
    fairly overcome with heat, undressed and bathed in a stream that lay in his path.

\section{Sutlun-Ex Romanised Text}

    Fusmapsut xel mapfalfuk nypet úkes muje. Kunjun nyket. Kunjun xylet falkul.\\
    Úk napsynet ene, jun lyk kunjune uwu, jun máj.\\
    Fusmapsut jusytet awa, Kunjun jukylet uke. Fusmapsut eksytet.
    Mapfalfuk jufylet. Kunjun lyket ukes falkule.\\
    Fusmapsut nymlunsytet ene, mapfalfuk máj.\\

\section{Sutlun-Ex Orthograpy}
\ortho{
    . Fusmapsut xel mapfalfuk nypet úkes muje . kunjun nyket.kunjun xylet falkul .\\
    . úk napsynet ene , jun lyk kunjune uwu , jun máj .\\
    . Fusmapsut jusytet awa , kunjun jukylet uke . Fusmapsut eksytet .\\
    . mapfalfuk jufylet . kunjun lyket ukes falkule .\\
    . Fusmapsut nymlunsytet ene , mapfalfuk máj .\\
}


\chapter{The Tower of Babel}
\section{English Text}

    And the whole earth was of one language, and of one speech.
    And it came to pass, as they journeyed from the east, that they found a plain in the land of Shinar;
    and they dwelt there.
    And they said one to another, Go to, let us make brick, and burn them throughly.
    And they had brick for stone, and slime had they for morter.
    And they said, Go to, let us build us a city and a tower, whose top may reach unto heaven;
    and let us make us a name, lest we be scattered abroad upon the face of the whole earth.
    And the LORD came down to see the city and the tower, which the children of men builded.
    And the LORD said, Behold, the people is one, and they have all one language; and this they begin to do:
    and now nothing will be restrained from them, which they have imagined to do.
    Go to, let us go down, and there confound their language, that they may not understand one another's speech.
    So the LORD scattered them abroad from thence upon the face of all the earth: and they left off to build the city.

    Therefore is the name of it called Babel; because the LORD did there confound the language of all the earth:
    and from thence did the LORD scatter them abroad upon the face of all the earth.

\section{Sutlun-Ex Romanised Text}

    Sujfus xylet xam sutlun. \\
    En fyket, úk niket jupem. Mukfusem úk myf ene, wamfus ((Sinal))es. Úk mukfusem jynet. \\
    Úk sutlynet ene úkem, úk týlej wamlume, úk jufýlej wamlume. \\
    Úk xylet wamlume lumem, úk xylet wakmule wamlumem. \\

\section{Sutlun-Ex Orthography}
\ortho{

    . sujfus xylet xam sutlun . \\
    . en fyk , úk niket jupem . mukfusem úk myfet ene , wamfus ((sinal))em . úk mukfusem jynet . \\
    . úk sutlynet ene úkem , úk týlej wamlume , úk jufýlej wamlume . \\
    . úk xylet wamlume lumem , úk xylet wakmule wamlumem . \\

}

\chapter{Chapter I of Epictetus' Enchiridion}
\section{English Text}

    There are things which are within our power, and there are things which are beyond our power.
    Within our power are opinion, aim, desire, aversion, and, in one word, whatever affairs are our own.
    Beyond our power are body, property, reputation, office, and, in one word,
    whatever are not properly our own affairs.

    Now the things within our power are by nature free, unrestricted, unhindered;
    but those beyond our power are weak, dependent, restricted, alien. Remember, then,
    that if you attribute freedom to things by nature dependent and take what belongs to others for your own,
    you will be hindered, you will lament, you will be disturbed, you will find fault both with gods and men.
    But if you take for your own only that which is your own and view what belongs to others just as it really is,
    then no one will ever compel you, no one will restrict you; you will find fault with no one,
    you will accuse no one, you will do nothing against your will; no one will hurt you, you will not have an enemy,
    nor will you suffer any harm.

    Aiming, therefore, at such great things, remember that you must not allow yourself any inclination,
    however slight, toward the attainment of the others; but that you must entirely quit some of them,
    and for the present postpone the rest. But if you would have these, and possess power and wealth likewise,
    you may miss the latter in seeking the former; and you will certainly fail of that by which alone
    happiness and freedom are procured.

    Seek at once, therefore, to be able to say to every unpleasing semblance, “You are but a semblance and by
    no means the real thing.” And then examine it by those rules which [18] you have; and first and chiefly by this:
    whether it concerns the things which are within our own power or those which are not; and if it concerns
    anything beyond our power, be prepared to say that it is nothing to you.

\section{Sutlun-Ex Romanised Text}

    <TO-DO>

\section{Sutlun-Ex Orthography}
\ortho{

    . fýkej .

}

\end{document}