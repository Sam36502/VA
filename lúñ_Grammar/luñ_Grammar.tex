\documentclass{book}
\usepackage[utf8]{inputenc}
\usepackage{tipa}

\newcommand{\entry}[4]{\textbf{#1} --- /\textipa{#2}/ \emph{#3} $\bullet$ #4\\}
\newcommand{\ö}{\textscoelig}

\begin{document}

\begingroup
\centering
\vfill
\Huge{REFERENCE \\ GRAMMAR}\\
\huge{\&}\\
\Huge{DICTIONARY}\\
\huge{of}\\
\Huge{Luñ}\\
\Large{Redo Title Page at some point...}\\
\large{By Samuel Pearce}\\
\vfill\null
\endgroup
\thispagestyle{empty}

\tableofcontents
\pagebreak

\section{Introduction}
Luñ is a constructed language, or conlang, which was created in 2021 for a school
assignment as an experiment. The goal was to create a language which was as small and as
easy to learn as possible. It began as a series of small languages, each being smaller
than the last. I was looking for the point where a language becomes impractically small
and no longer useful in day-to-day life. Luñ is the final result of this.



\part{Grammar}
\chapter{Phonology}
\section{Consonants}
\begin{center}
    \begin{tabular}{l|c|c|c}
                    & Bilabial          & Alveoalar  & Palatal \\
        \hline
        Nasal       & m                 & n         & \textipa{N} \\
        Plosive     & p                 & t         & k \\
        Fricative   & \textipa{F} [f]   & s         & x \\
        Liquid      & w                 & l         & j \\
    \end{tabular}
\end{center}

\section{Vowels}
\begin{center}
    \begin{tabular}{l|c|c|c}
                    & Front         & Centre            & Back \\
        \hline
        Close       & i y           &                   & \textipa{W [2]} u \\
        Middle      &               & \textipa{@ [e]}   & \\
        Open        & a \ö [\oe]    &                   & \\
    \end{tabular}
\end{center}

\section{Phonotactics}
In Luñ, roots are bi-consonantal and the vowel determines what part of speech the word is.
For these root words, the consonant structure is \textbf{CVC} Where V is any vowel except /\textipa{@}/,
C is any consonant.

\section{Orthography}

\subsection{Romanisation}
The romanisation used might seem quite strange to an outside observer, but it was designed to
emphasize the duality of the main vowels (y, u, a) with their rounded equivalents (ý, ú, á) which
represents a change in meaning for the roots. Though given that this might be difficult to understand
and not as easy to type as it is on a QWERTZ keyboard, a more phonetic alternative is also provided with
digraph alternatives to the diacritics used.

\begin{center}
    \begin{tabular}{|c|c|c|}
        \hline
        IPA & Rom. & Alt. \\
        \hline
        p           & p & p \\
        t           & t & t \\
        k           & k & k \\
        m           & m & m \\
        n           & n & n \\
        \textipa{N} & ñ & ng \\
        \textipa{F} & f & f \\
        s           & s & s \\
        x           & x & x \\
        \textipa{@} & e & e \\
        \hline
    \end{tabular}
    \begin{tabular}{|c|c|c|}
        \hline
        IPA & Rom. & Alt. \\
        \hline
        w           & w & w \\
        l           & l & l \\
        j           & j & j \\
        a           & a & a \\
        \ö          & á & ö/oe \\
        i           & y & i \\
        y           & ý & ü/ue \\
        \textipa{W} & u & u \\
        u           & ú & ú/uu \\
                    &   &  \\
        \hline
    \end{tabular}
\end{center}

\subsection{Writing System}
Given the rigidly structured syllables, I experimented with the idea of writing systems that used this
to their advantage for more regular and compact glyphs, but found this too complicated and received
feedback that confirmed this fear. So I decided to go for a simpler alphabetic system for the writing
system. I definitely wanted to make it a featural system though, because I had layed the phonemes out
in a systematic manner for this purpose.


\chapter{Morphology}
\section{Universal Inflections}
These are a few inflections (mostly prefixes) which can be applied to any root, no matter the
part of speech. Though these changes may not always yield a result that fully makes sense.

\subsection{Opposites}
You can form the opposite meaning of a 

\section{Nouns}
\subsection{Number}
In Fluñ, Nouns all have the "u" sound in the root which is unrounded for singular and rounded for plural.
For example:

\begin{center}
    "Mun" $\rightarrow$ "a game" \\
    "Mún" $\rightarrow$ "many games"
\end{center}

\subsection{Case}
Fluñ has 4 grammatical cases which are all formed with a simple suffix according to the following table:

\begin{center}
    \begin{tabular}{|r|l|l|}
        \hline
        Case Name   & Suffix    & Example \\
        \hline
        Nominative  & -         & pux \\
        Accusative  & -e        & puxe \\
        Dative      & -em       & puxem \\
        Genitive    & -es       & puxes \\
        \hline
    \end{tabular}
\end{center}


\section{Verbs}
\subsection{Mood}
Fluñ has two verb moods: Indicative \& Imperative. These are also formed by the root-sound's roundness.
All Verbs use the "y" sound for their roots. "Y" is indicative, while "ý" is imperative:

\begin{center}
    "ut kyñ" $\rightarrow$ "You go." / "You are going." \\
    "ut kýñ" $\rightarrow$ "You, go!"
\end{center}

\subsection{Tense}
Fluñ has 3 tenses which are all formed with a simple suffix according to the following table:

\begin{center}
    \begin{tabular}{|r|l|r|l|}
        \hline
        Tense Name  & Suffix    & Example   & Meaning \\
        \hline
        Past        & -et       & pixet     & ate, were eating \\
        Present     & - (-ef)   & pix       & eat, are eating \\
        Future      & -ej       & pixej     & will eat \\
        \hline
    \end{tabular}
\end{center}
The present tense is the default tense and needn't be marked, but if it is, it emphasizes that
the action is taking place now. E.g.:

\begin{center}
    "ut kyñ kumem?" $\rightarrow$ "Where are you going?" \\
    "ut kyñef kumem?" $\rightarrow$ "Where are you going now?"
\end{center}


\section{Adjectives}
\subsection{Positive \& Superlative}
Adjectives in Fluñ all have the "a" sound in their root which is rounded to form the superlative
form of the adjective.

\begin{center}
    "tas pux" $\rightarrow$ "good food" \\
    "tás pux" $\rightarrow$ "the best food"
\end{center}

\subsection{Comparing}
To compare two things, you can add the augmentitive prefix to the greater adjective with a 


\chapter{Syntax}

\chapter{Sentence Order}



\part{Lexicon}

\begin{center}
    \Huge{P}
\end{center}

{\Large\textbf{p-x}} \\
\entry{pux}{"p2x}{n. sg.}{Food, an item of food, a meal}
\entry{púx}{"pux}{n. pl.}{Food, many items of food}
\entry{pyx}{"pix}{v. ind.}{to eat, to drink, to consume}
\entry{pýx}{"pyx}{v. imp.}{eat!, drink!, consume!}
\entry{pax}{"pax}{a. pos.}{edible}
\entry{páx}{"p\ö x}{a. sup.}{most edible}

\begin{center}
    \Huge{T}
\end{center}

\begin{center}
    \Huge{K}
\end{center}

{\Large\textbf{k-ñ}} \\
\entry{kuñ}{"k2N}{n. sg.}{a walk, a motion/movement, a journey}
\entry{kúñ}{"kuN}{n. pl.}{many walks, many motions/movements, many journeys}
\entry{kyñ}{"kiN}{v. ind.}{to walk, to move, to go}
\entry{kýñ}{"kyN}{v. imp.}{walk!, move!, go!}
\entry{kañ}{"kaN}{a. pos.}{in motion, moving, going, living}
\entry{káñ}{"k\ö N}{a. sup.}{moving the most, the most alive}

\end{document}
